
\documentclass[12pt]{article}
\usepackage[margin=1in]{geometry}
\usepackage{amsmath}
\usepackage{amssymb}
\usepackage{hyperref}
\usepackage{enumitem}
\setlength{\parskip}{0.8em}

\title{Cardinality-Constrained Knapsack: Full Problem-Solving Journal}
\author{}
\date{}

\begin{document}

\maketitle

\section*{1. Initial Problem Recognition and Confusion}

You recognized a problem scenario (“3 bananas, 2 diamonds, 4 bags of sand, find an optimal configuration for 3 items”) that \textit{resembles knapsack} in some sense. You noted it felt like “coin change II” but with limited availability.

\subsection*{Is it just 0/1 knapsack? Or is there something else, like cardinality constraints?}

You initially wondered whether “cardinality-constrained” just meant each item is 0/1.

\textbf{Clarification:} “Cardinality-constrained” means you can pick \textbf{at most $K$ items} overall, not just that each item is 0/1. In plain 0/1 knapsack, you only worry about the capacity $W$. Here, you must also ensure the \textit{count} of chosen items doesn’t exceed $K$.

\section*{2. Defining the Recursive Structure (Take/Skip)}

\subsection*{You Proposed}

A recursion with two calls for each item:
\begin{itemize}
  \item \textbf{Skip:} move to $i+1$ without changing weight or $K$
  \item \textbf{Take:} move to $i+1$, reduce remaining weight by the item’s weight, reduce $K$ by 1, and add the value
\end{itemize}

You recognized this was a standard “take or skip” pattern used in knapsack.

\subsection*{Clarification}

\begin{itemize}
  \item If $w < 0$ or $k < 0$, return $-\infty$ to mark that path as invalid.
  \item Once you've processed all items ($i == n$), return 0.
  \item Memoization is essential to prevent exponential explosion.
\end{itemize}

\section*{3. Defining DP and the Resulting Complexity}

You defined a state like:

\[
\texttt{dp[i][k][w]} = \text{maximum dollars you can collect starting at item } i \text{ with } k \text{ picks and weight capacity } w
\]

If weights range up to $n^3$, and the total capacity $W$ is up to $n^4$, then the number of potential weight states becomes very large.

So the total number of DP states is:

\[
n \times n \times n^4 = O(n^6)
\]

\textbf{Each state is computed once.} Even if many states are infeasible and return $-\infty$, they still need to be filled.

\subsection*{Misconception / Correction}

\begin{itemize}
  \item You thought time might be $O(n^2)$ due to double recursion.
  \item \textbf{Correction:} Each DP state requires constant time, and there are $O(n^6)$ of them.
\end{itemize}

\section*{4. Comparisons to Other DP Patterns}

\subsection*{Tower of Hanoi}

You recalled that Tower of Hanoi’s double recursion goes from $O(n)$ to $O(2^n)$. 

\textbf{Clarification:} With memoization, the cost is based on number of states, not branching factor.

\subsection*{Coin Change II}

You compared it to “Coin Change II” — which is an \textbf{unbounded} knapsack. This is more like 0/1 knapsack \textit{plus} a total item-limit constraint.

\section*{5. Indicators This Problem is “Knapsack Land”}

\begin{itemize}
  \item Items with weight and value? \textbf{Yes}
  \item Total capacity constraint? \textbf{Yes}
  \item Want to maximize value? \textbf{Yes}
  \item Take-or-skip decisions? \textbf{Yes}
  \item Extra constraint like $\leq K$ items? \textbf{Yes}
\end{itemize}

\section*{6. Key Observations and Corrections}

\begin{itemize}
  \item \textbf{Observation:} “Cardinality-constrained might just mean 0/1?”
  
    \textbf{Correction:} It means at most $K$ items — stricter than 0/1.
  
  \item \textbf{Observation:} “Naive recursion is $O(2^n)$, memoized is maybe $O(n^2)$?”

    \textbf{Correction:} Number of states is $O(n^6)$ due to large weight bounds.

  \item \textbf{Observation:} “Returning $-\infty$ prunes search?”

    \textbf{Clarification:} It just invalidates the value. The state is still filled in the DP table.

  \item \textbf{Observation:} “Tower of Hanoi-style double recursion applies?”

    \textbf{Clarification:} Not in DP with memoization — cost comes from total number of unique states.
\end{itemize}

\section*{7. Big Takeaways}

\begin{itemize}
  \item The state is defined as \texttt{dp[i][k][w]} — starting at item $i$, with $k$ items left, and $w$ remaining capacity.
  \item “Take” moves to $i+1$, decreases $k$ and $w$, and adds value.
  \item “Skip” moves to $i+1$ without changing $k$ or $w$.
  \item Base case: $k < 0$ or $w < 0$ returns $-\infty$; $i == n$ returns 0.
  \item Total time complexity is $O(n^6)$ in worst case.
  \item Invalid states still count toward worst-case bounds.
  \item Recognize patterns of knapsack when you see:
    \begin{itemize}
      \item Items
      \item Capacity constraint
      \item Value to maximize
      \item Take/skip logic
      \item Cardinality limits
    \end{itemize}
\end{itemize}

\end{document}
